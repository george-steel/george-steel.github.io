% !TEX program = xelatex
\documentclass[11pt,letterpaper,sans]{moderncv}

\usepackage{fontspec}
\usepackage{xltxtra}
\usepackage{xunicode}
\usepackage{xcolor}
\usepackage[tt=false]{libertine} %set font
\newfontface\monopropfont[Scale=MatchLowercase]{Latin Modern Mono Prop}
\def\UrlFont{\monopropfont}

\definecolor{myblue}{cmyk}{1,0.4,0,0.2}
\moderncvstyle{classic}                        % style options are 'casual' (default), 'classic', 'oldstyle' and 'banking'

%\moderncvcolor{blue}                          % color options 'blue' (default), 'orange', 'green', 'red', 'purple', 'grey' and 'black'
\definecolor{color0}{cmyk}{0,0,0,1}
\definecolor{color1}{cmyk}{1,0,1,0.2}
\definecolor{color2}{cmyk}{1,0.4,0,0.2}

%\nopagenumbers{}                             % uncomment to suppress automatic page numbering for CVs longer than one page

\usepackage[margin=6pc,letterpaper,top=4pc,left=4pc,bottom=5pc]{geometry}
\setlength{\hintscolumnwidth}{5pc}

\sloppy

\usepackage{url}

% personal data
\firstname{George}
\familyname{Steel}
%\title{Resumé}                          % optional, remove / comment the line if not wanted
\address{14 Cypress Ct.}{Aurora, ON, L4G 6S8}    % optional, remove / comment the line if not wanted
\phone{647-771-5414}                     % optional, remove / comment the line if not wanted
\email{george.steel@gmail.com}                          % optional, remove / comment the line if not wanted
%\social[github]{github.com/george-steel}
\collectionadd[github]{socials}{\protect\httplink[github.com/george-steel]{www.github.com/george-steel}}

\pretolerance=10000
\tolerance=2000 
\emergencystretch=10pt




\begin{document}
\makecvtitle

\vspace{-1.5pc}
\sloppy


\section{Technologies and Languages}
\cvitem{Languages}{Go, Rust, C++, C, JavaScript, Typescript, Haskell, Python, SQL, \TeX}
\cvitem{Technologies}{FIDO2/Webauthn, MFA, OAuth, Lit, WebAnimations, Blink, WPT testing, W3C specs, Postgres, CockroachDB, GTK, EPUB, Flask}


\section{Education}

\cventry{2010\,--\,2016}{University of Toronto}{Hon.\ B.\,Sc.\ in Mathematics}{}{}{\begin{itemize}
    \item Minors in Computer Science and Biology.
    \item Graduated with High Distinction (GPA 3.9).
\end{itemize}}

% \cventry{2005--2010}{Completed Aurora High School}{Completed Grade 12 concentrating in Maths and Sciences}{}{}{\begin{itemize}
%     \item Received straight A+ in grade 12
%     \item Finalist in POPTOR (U of T physics contest) in 2009, 2010.
%     \item 3th place nationally in Chem 13 News contest (Waterloo) in 2010. 8th place in 2009.
%     \item 3rd place nationally in Avogadro chemistry contest (Waterloo) in 2008.
%     \item 1st place nationally in Canadian Computing Competition, Junior division, 2007.
% \end{itemize}}



\section{Employment History}

\cventry{Apr 2021\,--\\Jun 2023}{Software Developer}{}{LoginID Canada}{}{\begin{itemize}
    \item Created an OAuth2.1/OpenID Connect identity provider supporting configurable multi-factor authentication methods, including passwordless FIDO2, SMS, TOTP, and others.
    \item Led multiple refactoring projects to reduce tech debt and improve performance in related services, eventually architecting a move to a single-binary to reduce microservice overhead as well as helping to architect a rewrite of our core FIDO2 library in go.
    \item Created a custom high-performance database client library for our Go services based on sqlx and pgxpool. This allowed queries to be performed in all transactional modes supported by CockroachDB (including the fast batch mode unsupported by database/sql and existing ORMs) with the results automatically mapped into slices of structs.
    \item Led code reviews for a large portion of our software stack. 
\end{itemize}}

\cventry{Jun 2019\,--\\Nov 2020}{Software Developer}{Chrome Animations/Interactions}{Google Canada}{}{\begin{itemize}
    \item Helped complete and launch the WebAnimations API in Blink (the rendering and Javascript engine powering Chrome and Electron used on over 3 billion devices), giving a unified Javascript interface to declarative animations from all sources using an invalidatable keyframe model.
    \item Contributed a section to the WebAnimations spec (including cross-browser Web Platform Tests) allowing the creation and manipulation of animations targeting pseudo-elements.
    \item Contributed numerous bugfixes and optimizations across the Blink animations stack (written in C++). This included allowing percentage transform animations (sidebars and some popups) to run off-thread, allowing them to run more smoothly despite the actions of other scripts on the same page.
    \item Participated in the Chrome Interactions bug triage rotation, which is the first point of contact for game developers (and others) dealing with possible engine problems.
    \item Contributions at \url{https://chromium-review.googlesource.com/q/owner:gtsteel@chromium.org}
\end{itemize}}

\cventry{Nov 2017\,--\\Feb 2019}{Full-stack Software Developer}{Satsuma Labs}{}{}{\begin{itemize}
    \item Created a prototype mobile application using a Haskell backend and a React-Native frontend.
    \item Developed a number of open-source libraries furthering the Haskell web service and react-native ecosystems, including a spatial indexing layer which we used with CockroachDB.
    (Available at \url{https://github.com/SatsumaLabs})
\end{itemize}}

\cventry{May 2016\,--\\May 2017}{Software Developer}{Prof.\ Peter Jurgec}{Linguistics, University of Toronto}{}{\begin{itemize}
    \item Created browser-based educational software used in introductory phonology courses.
    \item Rewrote and further developed a research tool which uses a maximum-entropy machine learning model to analyze the relative frequency of sound patterns in speech based on sample text and generate random pronounceable gibberish based in the inferred constraints.
\end{itemize}}

% \cventry{Sept 2014\,--\\May 2016}{Private tutoring}{}{}{}{
%     Helped a student overcome her difficulties and pass multiple specialist-level math courses she had failed on her previous attempts.}

\cventry{Sept 2013\,--\\Mar 2014}{IT Assistant}{ENAGB Youth Program}{Native Canadian Centre of Toronto}{}{
    Created a responsive website for the ENAGB program (featuring a dynamic events calendar) along with a variety of promotional materials (posters, brochures, business cards, etc.) for the program.
}

\cventry{Summer 2011,\\2012}{Summer Research Assistant}{Prof.\ Gilbert Walker}{Chemistry, University of Toronto}{}{
    Performed spectroscopy and microscopy supporting research into creating nanoparticle based markers for medical diagnostic use, improving the sensitivity non-destructive procedures for determining particle shape.
}

\cventry{Summer 2010}{Intern}{Kerr Vayne Systems}{}{}{
    Created a web application to stream real-time data for schedule display in a broadcast automation system.
    }

% \cventry{Sept 2009--Jan 2010}{Intern}{Zymurgy Systems Inc.}{}{}{\begin{itemize}
%     \item Worked on further developing an existing web content management system and creating a new project.
%     \item Learned the basics of the HTTP protocol and PHP/HTML/AJAX/JQuery development.
% \end{itemize}}



% \section{Community Involvement}

% \cventry{Spring 2013}{Intern}{Teaching Drum Outdoor School}{}{}{\begin{itemize}
%     \item Assisted with several outdoor projects including the spring leek harvest and repairing several buildings.
%     \item Set up and built equipment for a recording studio and CD production operation.
%     \item Completed all design and production work for a CD release.
% \end{itemize}}

% \cventry{2012--2013}{Founding Member of Advisory Council}{ENAGB Youth Program}{Native Canadian Centre of Toronto}{}{
%     Helped to guide the formation of a new mental health program dedicated to serving Native youth in Toronto, which I stayed involved with for many years afterwards.}



\section{Releases and Publications}
\cventry{2024}{Bowfishing Blitz}{}{}{\url{https://github.com/george-steel/bowfishing-blitz}}{
    A tech demo/minigame which demonstrates my new graphical technique of clip-space planar refraction, used to render water with physically-accurate refraction without performance costs of raytracing.
}
\cventry{2021\,--\,present}{Contributions to Ink/Stitch}{}{}{\url{https://github.com/inkstitch/inkstitch}}{
    A tool used to program numerically-controlled embroidery machines. 
    I added a new algorithm for running stitch along curves which gets much more uniform stitch spacing as well as a system for randomized satin stitch which stays stable under changes in path shape.
}
\cventry{2021}{Ojibwe Dictionary App}{}{}{\url{https://ojibwe-dict-test.netlify.app/}}{
    A searchable interface to the Ojibwe-English dictionary compiled by Weshki-ayaad, Charlie Lippert, and Guy Gambill.
    Used Rust and WebAssembly to create an approximate search function that works effectively despite local spelling differences.
}
\cventry{2019}{persistent-spatial}{}{}{\url{https://hackage.haskell.org/package/persistent-spatial}}{
    A data structure for storing and indexing geographic coordinates which can be used with any SQL database to provide fast area queries using a standard b-tree index. Originally used with CockroachDB, which had no spatial index support at the time of release.
}
\cventry{2017}{Maxent Phonotactic Learner}{}{}{\url{https://github.com/george-steel/maxent-learner}}{
    A machine-learning tool for automatically inferring phonotactic grammars from a lexicon and using those grammars to generate random text, based on Hayes and Wilson's \textit{A Maximum Entropy Model of Phonotactics and Phonotactic Learning}.
}
\cventry{2017}{frpnow-gtk3}{}{}{\url{https://hackage.haskell.org/package/frpnow-gtk3}}{
    High-level interface for GTK3 using FRPNow for asynchronous, reactive event handling.
}
\cventry{2016\,--\,2017}{PhonoApps}{with Prof.\ Peter Jurgec}{}{\url{http://phonology.us/}}{
    Computational and learning tools for phonologists.
}

\end{document}
